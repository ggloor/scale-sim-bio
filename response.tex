% Options for packages loaded elsewhere
\PassOptionsToPackage{unicode}{hyperref}
\PassOptionsToPackage{hyphens}{url}
%
\documentclass[
]{article}
\usepackage{amsmath,amssymb}
\usepackage{iftex}
\ifPDFTeX
  \usepackage[T1]{fontenc}
  \usepackage[utf8]{inputenc}
  \usepackage{textcomp} % provide euro and other symbols
\else % if luatex or xetex
  \usepackage{unicode-math} % this also loads fontspec
  \defaultfontfeatures{Scale=MatchLowercase}
  \defaultfontfeatures[\rmfamily]{Ligatures=TeX,Scale=1}
\fi
\usepackage{lmodern}
\ifPDFTeX\else
  % xetex/luatex font selection
\fi
% Use upquote if available, for straight quotes in verbatim environments
\IfFileExists{upquote.sty}{\usepackage{upquote}}{}
\IfFileExists{microtype.sty}{% use microtype if available
  \usepackage[]{microtype}
  \UseMicrotypeSet[protrusion]{basicmath} % disable protrusion for tt fonts
}{}
\makeatletter
\@ifundefined{KOMAClassName}{% if non-KOMA class
  \IfFileExists{parskip.sty}{%
    \usepackage{parskip}
  }{% else
    \setlength{\parindent}{0pt}
    \setlength{\parskip}{6pt plus 2pt minus 1pt}}
}{% if KOMA class
  \KOMAoptions{parskip=half}}
\makeatother
\usepackage{xcolor}
\usepackage[margin=1in]{geometry}
\usepackage{graphicx}
\makeatletter
\newsavebox\pandoc@box
\newcommand*\pandocbounded[1]{% scales image to fit in text height/width
  \sbox\pandoc@box{#1}%
  \Gscale@div\@tempa{\textheight}{\dimexpr\ht\pandoc@box+\dp\pandoc@box\relax}%
  \Gscale@div\@tempb{\linewidth}{\wd\pandoc@box}%
  \ifdim\@tempb\p@<\@tempa\p@\let\@tempa\@tempb\fi% select the smaller of both
  \ifdim\@tempa\p@<\p@\scalebox{\@tempa}{\usebox\pandoc@box}%
  \else\usebox{\pandoc@box}%
  \fi%
}
% Set default figure placement to htbp
\def\fps@figure{htbp}
\makeatother
\setlength{\emergencystretch}{3em} % prevent overfull lines
\providecommand{\tightlist}{%
  \setlength{\itemsep}{0pt}\setlength{\parskip}{0pt}}
\setcounter{secnumdepth}{-\maxdimen} % remove section numbering
% definitions for citeproc citations
\NewDocumentCommand\citeproctext{}{}
\NewDocumentCommand\citeproc{mm}{%
  \begingroup\def\citeproctext{#2}\cite{#1}\endgroup}
\makeatletter
 % allow citations to break across lines
 \let\@cite@ofmt\@firstofone
 % avoid brackets around text for \cite:
 \def\@biblabel#1{}
 \def\@cite#1#2{{#1\if@tempswa , #2\fi}}
\makeatother
\newlength{\cslhangindent}
\setlength{\cslhangindent}{1.5em}
\newlength{\csllabelwidth}
\setlength{\csllabelwidth}{3em}
\newenvironment{CSLReferences}[2] % #1 hanging-indent, #2 entry-spacing
 {\begin{list}{}{%
  \setlength{\itemindent}{0pt}
  \setlength{\leftmargin}{0pt}
  \setlength{\parsep}{0pt}
  % turn on hanging indent if param 1 is 1
  \ifodd #1
   \setlength{\leftmargin}{\cslhangindent}
   \setlength{\itemindent}{-1\cslhangindent}
  \fi
  % set entry spacing
  \setlength{\itemsep}{#2\baselineskip}}}
 {\end{list}}
\usepackage{calc}
\newcommand{\CSLBlock}[1]{\hfill\break\parbox[t]{\linewidth}{\strut\ignorespaces#1\strut}}
\newcommand{\CSLLeftMargin}[1]{\parbox[t]{\csllabelwidth}{\strut#1\strut}}
\newcommand{\CSLRightInline}[1]{\parbox[t]{\linewidth - \csllabelwidth}{\strut#1\strut}}
\newcommand{\CSLIndent}[1]{\hspace{\cslhangindent}#1}
\usepackage{setspace}
\usepackage{bookmark}
\IfFileExists{xurl.sty}{\usepackage{xurl}}{} % add URL line breaks if available
\urlstyle{same}
\hypersetup{
  pdftitle={Response to Reviewers},
  hidelinks,
  pdfcreator={LaTeX via pandoc}}

\title{Response to Reviewers}
\author{true}
\date{}

\begin{document}
\maketitle

Reviewer: 1

Comments to the Author

Gloor, Nixon, and Silverman provide a clear and concisely written
manuscript that describes a significant improvement of the ALDEx2
package to incorporate scale uncertainty in analyses of differential
expression and abundance. These scale models relieve the reliance on
commonly used but difficult to rationalize cutoffs of q-values and
log-fold change thresholds, and facilitate adoption through the ALDEx2 R
package. This work represents a valuable addition to an already
impactful and versatile tool for the community. I have several minor
comments that I am confident can be easily addressed by the authors.

\begin{enumerate}
\def\labelenumi{\arabic{enumi}.}
\item
  The authors focus here on the application of scale-informed ALDEx2 to
  transcriptomic analysis. Can the authors further discuss whether
  applications to other types of measurements is equally appropriate? If
  so, there may be an opportunity to generalize the title to use
  ``counts'' instead of ``RNA-sequencing'', as that broadens
  applicability to other data types (e.g.~proteomics, single cell
  epigenome data, etc).

  \begin{itemize}
  \tightlist
  \item
    We have changed the title to reflect that it is count data, and have
    included information on other data types in the introduction.
    However, the focus of this report is intended to be squarely on
    transcriptome and metatranscriptome data in an attempt to more
    tightly target that analytic audience. Companion articles by by
    McGovern et al. (1) discussing the use in gene set enrichment and by
    Nixon et al. (2) targeting the microbiome community are published in
    PLoS Comp. Bio. and under second revision at Genome Biol.
    respectively.
  \end{itemize}
\item
  Relevant to statements like this: ``The original naive ALDEx2 (39)
  model unwittingly made a strict assumption about scale through the CLR
  normalization (3).'' - Can ``scale-naïve'' be used instead to more
  precisely indicate the distinction between the previous and current
  defaults in ALDEx2?

  \begin{itemize}
  \tightlist
  \item
    We thank the reviewer for this suggestion of more precise and
    descriptive terminology.
  \end{itemize}
\item
  Why this threshold in Figure 1? ``The horizontal dashed lines
  represent a log2 difference of ±1.4'' The text goes on to say the
  following: (line 41-44, pg 6): ``Note that there is considerable
  variation in recommended cutoff values(14). Here, applying a
  dual-cutoff using a heuristic of at least a 2\^{}1.4 fold change
  reduces the number of significant outputs to 193 for DESeq2 and to 186
  for ALDEx2. This cutoff was chosen for convenience and is in-line with
  the recommendations of (14) with the fold change limits shown by the
  dashed grey lines in Figure 1.'' - Please add a rationale for
  selection of this cutoff, and briefly state the guideline used here,
  especially given the variation in recommendations in ref 14.

  \begin{itemize}
  \tightlist
  \item
    We agree that the explanation for this cutoff was not well
    explained, and now include a section in the results explaining that
    all fold-change cutoffs are arbitrarily chosen and this one was
    chosen to be comparable in number of transcripts identified to our
    gamma parameter. There is now an example in the text (Figure 3) and
    in the supplement (Sup. Figure 2) showing that adding scale
    uncertainty is superior in controlling FDR when compared to
    fold-change cutoffs.(lines 19-20, 160-169)
  \end{itemize}
\item
  Supplementary Fig 1 is an excellent demonstration of how scale
  uncertainty affects the number of DE genes -- the recommendation in
  that section that this is be used a diagnostic plot should be
  emphasized in the main text as well, possibly on line 50-51, or in the
  discussion.

  \begin{itemize}
  \tightlist
  \item
    Thank you we have enlarged the discussion of this and included a
    second example using a biologically replicated dataset with modeled
    TP information.
  \end{itemize}
\item
  This dataset (ref 44) included many technical replicates. ~How do
  these conclusions port to datasets that have biological replicates
  instead?

  \begin{itemize}
  \tightlist
  \item
    we have included two worked examples that use biological datasets
    from human-derived studies along with modeled TP data. (lines
    177-237)
  \end{itemize}
\item
  Figure 2B, and pg 9 lines 11-14: ``Figure 2B shows a plot of the
  difference between the \(\gamma = 0\) and \(\gamma = 1\) data and here
  we can see that scale uncertainty is preferentially increasing the
  dispersion of the mid-expressed transcripts that formerly had
  negligible dispersion; examine the grey line of best fit (overlaid by
  the red line) for the trend.'' - Can the 5 lines be separately shown
  on 5 panels, rather than overlaid? They are difficult to visually
  distinguish as presented.

  \begin{itemize}
  \tightlist
  \item
    We realize that this figure was confusing in concept, and have moved
    it to the supplement as Supp. Figure 5. This figure is now only
    discussed in support of figures that we believe better make the
    point that there is not an across-the-board change to dispersion and
    hence p-values. This is now better described by Figure 2 panels C
    and D, and Supp. Figure 3 panels C and D. This information is now
    only briefly discussed in lines 243-245, and is no longer central to
    our argument.
  \end{itemize}
\item
  Is the conclusion from the first part of the results that by adding
  scale uncertainty (\(\gamma =1\)), one can do without setting a change
  threshold (T=1.4 in this case). ~This should be emphasized in the
  discussion.

  \begin{itemize}
  \tightlist
  \item
    Yes, and the inclusion of the modeled data now makes this point
    explicitly. The discussion has been simplified to more clearly
    highlight this point. We are explicit in lines 246-251, 367-374 and
    381-388
  \end{itemize}
\item
  The examples outlined in the results would benefit from a more
  explicit description of key experimental features -- e.g.~the number
  of technical replicates in the ref 44 dataset (mentioned in
  discussion, but would be useful in results); same for the ref 50 data,
  so the reader can easily contextualize the results, and draw parallels
  to their datasets of interest to which ALDEx2 may be applicable.

  \begin{itemize}
  \tightlist
  \item
    We have included more description of the datasets and made clear
    what is technical and what is biological replication especially at
    lines 128-134, 177-185, 253-259
  \end{itemize}
\end{enumerate}

Some typos:

Line 25, pg 10, typo: ``functions tp be nearly invariant''

Grammatical error: This assumption was often close enough to the true
value to be useful, but was not always the a good estimate and could be
outperformed by other normalizations (40).

\begin{itemize}
\tightlist
\item
  Thank you, we have tried to catch all typos and grammatical errors.
\end{itemize}

Reviewer: 2

Comments to the Author Summary: In this manuscript Gloor et
al.~highlight the application of ALDEx2 and the recent addition of scale
models to help improve differential expression analysis in RNA seq
experiments. This manuscript builds upon previous work that introduced
scale models to ALDEx2 and highlights that differential analysis in high
throughput sequencing often suffers from scale assumptions introduced
during commonly applied normalization techniques. In this manuscript
they used both a transcriptomic and metatranscriptomic experiment to
highlight the improvements that can be achieved by using the scale
models incorporated in ALDEx2. The manuscript is well written and
highlights an important problem within the field, however, there are
some major comments that should be addressed by the authors.

Major comments:

\begin{enumerate}
\def\labelenumi{\arabic{enumi}.}
\item
  Throughout the manuscript a clear definition of what the ``truth'' is
  for each experimental set up is needed (i.e., is it transcripts
  relative to species abundance or bulk absolute transcripts or
  something else?). This is critically important in the
  metatranscriptomics section where high levels of differing biomass
  could result in higher levels of absolute transcripts across the board
  but not higher levels of transcripts relative to the underlying
  species abundance. The attempt to center the data on housekeeping
  genes suggests the latter is the objective but this is not clear
  throughout the manuscript. In addition to this if the authors are
  interested in identifying the transcript levels relative to species
  abundance, they may be interested in comparing their tests against a
  method that normalizes transcript levels against underlying DNA
  abundances(\url{https://pubmed.ncbi.nlm.nih.gov/34465175/}).

  \begin{itemize}
  \tightlist
  \item
    We have added two semi-synthetic datasets to show how the FDR is
    controlled much better by scale uncertainty than by fold-change
    cutoffs. Ultimately, the choice of what to use as the baseline for
    normalization of scale between samples is a choice that must be made
    by the analyst. In some situations the biological question may be
    best addressed by normalizing transcript levels to species abundance
    (although to be clear, in the method suggested both the RNA and
    species levels are relative abundances). However, such a
    normalization is discussed with reference to species-level
    metatranscriptomics rather than systems-level metatranscriptomics as
    shown in our report and in (3). We would need to redo all of the
    analysis in the cited paper in order to make this comparison.
    Indeed, the scale approach advocated in this report can use any
    measure of absolute abundance as an anchor and we have been explicit
    throughout about this. (lines 177-237, 377-380)
  \end{itemize}
\item
  The claims in Figure 1 should be further validated through simulated
  data and or a dataset where the true underlying number of
  differentially expressed transcripts is known. While the reduction in
  transcripts that the scale model achieves is likely inline with the
  biology it is unclear if the model is being overconservative.

  \begin{itemize}
  \tightlist
  \item
    We thank the reviewer for this suggestion while the original paper
    by Nixon et al. (4) included extensive simulation and theoretic
    justification we recognize that this is not yet published and a
    clear demonstration of the utility in transcriptome datasets would
    be helpful. We have now included two simulated datasets from a
    recent benchmarking paper {[}2Li:2022aa{]} showing that the
    scale-reliant approach is substantially better in controlling the
    FDR and, at least in one example, has minimal effect on power (Fig
    2, Supp Fig 3). This has resulted in a major re-write of the results
    section and has made a substantial improvement for how we understand
    scale uncertainty and its effect on data analys. We again thank the
    reviewer for pushing us to do this analysis. (lines 177-237)
  \end{itemize}
\item
  In the metatranscriptomics section it is claimed that biomass differs
  by 20 fold between the groups that are being compared, however, the
  scaled model only includes a difference of 14\%. Please address this
  discrepancy. Furthermore, based on Figure 3B, the gamma model by
  itself was ineffective in this setting, so the manuscript should
  clarify when only the gamma parameter versus the gamma and mu
  parameters must be informed.

  \begin{itemize}
  \tightlist
  \item
    We have tried to include more insights in the results (Supp Figure
    4) and the discussion around when to use an informed vs a simple
    scale model. In essence, the centre of mass of the difference
    between groups should be close to 0 (no difference) to avoid FP and
    FN (5). (lines 358-365, 398-410)
  \end{itemize}
\item
  Due to the limited number of datasets shown in this manuscript the
  robustness of the model for various biological and technical
  conditions is difficult to assess.
\end{enumerate}

\begin{itemize}
\tightlist
\item
  We have added two additional datasets showing generality. (see above)
\end{itemize}

\begin{enumerate}
\def\labelenumi{\arabic{enumi}.}
\setcounter{enumi}{4}
\tightlist
\item
  Data availability: Scripts used to run the analysis (beyond just the
  model itself) should be provided in a well-documented GitHub (or
  equivalent).
\end{enumerate}

\begin{itemize}
\tightlist
\item
  Thank you for pointing out this oversight. The entire analysis is on
  github and I forgot to flip this to public access.
\end{itemize}

Minor comments: 1. Without external information, the Lambda distribution
appears entirely determined by the prior. It is unclear how robust the
choice of Lambda is across differing HTS experiments. Fig. 3 seems to
provide a case in which lambda alone is not enough.

\begin{itemize}
\tightlist
\item
  We have added more guidance and examples on how to choose appropriate
  parameters as outlined in the response to reviewer 1. (lines 398-410)
\end{itemize}

\begin{enumerate}
\def\labelenumi{\arabic{enumi}.}
\setcounter{enumi}{1}
\tightlist
\item
  ``The computational methods developed for microbiome analysis have low
  correlation with the actual scale but are useful (8).'' Well this may
  be true it has been shown that in general coefficients from the
  relative scale correlate strongly with those from the absolute scale.
  Especially in cases where biomass is not significantly changing
  between groups. This should be addressed.
\end{enumerate}

\begin{itemize}
\tightlist
\item
  Unfortunately, this statement is only true in the context of the
  machine learning model in microbiome analyses. This statement is
  intended to point out that the inclusion of differences in the
  location in scale between groups \emph{coupled with uncertainty in
  scale} are robust. This is also the intent of the Supplementary Figure
  7 showing that 14\% and 5\% scale offset are essentially equivalent as
  long a some uncertainty is accepted.
\end{itemize}

\begin{enumerate}
\def\labelenumi{\arabic{enumi}.}
\setcounter{enumi}{2}
\tightlist
\item
  The authors claim that ``the dispersion in the unscaled analysis in
  Figure 2A reaches a minimum near the mid-point of the distribution''
  which ``makes the counterintuitive suggestion that the variance in
  expression of the majority of genes with moderate expression is more
  predictable than highly-expressed genes or of housekeeping genes.''
  However, the dispersion seems essentially constant beyond the
  midpoint, with potentially a small difference in the tail. Whether or
  not that is biologically important and to what degree that impacts
  traditional differential expression analysis is unclear and not
  presented in a convincing manner. I wouldn't say this small change in
  the tail is inconsistent with the notation that ~sufficiently
  expressed genes all have equally predictable expression.
\end{enumerate}

\begin{itemize}
\tightlist
\item
  we agree that this figure was not optimal for the points we were
  trying to make. This figure has been moved to the supplement and
  de-emphasized. As indicated in the response to reviewer 1, the point
  that the addition of scale uncertainty is affecting transcripts based
  on dispersion and not on difference between is now made much better by
  Figure 2 and Supplementary Figure 3. (lines 19-20, 160-169)
\end{itemize}

\begin{enumerate}
\def\labelenumi{\arabic{enumi}.}
\setcounter{enumi}{3}
\item
  In the third paragraph of the discussion the authors claim that their
  method can control type 1 and 2 errors robustly. However, the authors
  never directly present any evidence that the ``hits'' their tools are
  finding are better at controlling these errors than others when a
  known ground truth is present (they only show that they have reduced
  numbers of hits and that they may be more reasonable with the
  underlying biology). The authors may want to benchmark their work on
  previous simulated data from ref. 18 if they want to make this claim.

  \begin{itemize}
  \tightlist
  \item
    We have now included two modelled datasets to bolster this claim
    with biological data and modeled true positive (TP) transcripts.
  \end{itemize}
\item
  The clause ``wet-lab protocols only provide information on the size of
  the data downstream of the step in the sample preparation protocol
  where the intervention was made'' should be reviewed as previous works
  have shown that these methods can give relatively accurate log fold
  changes under real experimental conditions.

  \begin{itemize}
  \tightlist
  \item
    We agree that spike-ins can be useful, and indeed the use of full
    scale models is a general tool to do so. However, to give a trivial
    example, spiking in at the time the RNA library is made will not
    control for the number of cells in the environment.
  \end{itemize}
\item
  The authors should make it clear that their works are referring to the
  absolute and not compositional scale as some researchers may be
  interested in the composition of transcripts within their sample
  rather than attempting to infer absolute changes from that composition
  (which is what is highlighted within this works).

  \begin{itemize}
  \tightlist
  \item
    We have included additional information around what scale is and
    what we are trying to measure. Figure 1 is now included to make this
    clearer. (line 48 now states this explicitly and we give examples
    throughout regarding what absolute abundance information can be
    included)
  \end{itemize}
\item
  In Fig. 2, what is rAbundance referring to? Is it log2(percent
  abundance)?

  \begin{itemize}
  \tightlist
  \item
    this is now supplementary Figure 5 and the rAbundance term has been
    clarified
  \end{itemize}
\item
  The Fig. 3 caption is inconsistent with the legend in the top of 3A.
  The text caption seems to be correct, but this should be fixed.

  \begin{itemize}
  \tightlist
  \item
    We have added information to the legend in now Figure 4 that the
    first panel has 0 scale uncertainty.
  \end{itemize}
\item
  I'm not entirely convinced that low dispersion is to blame for issues
  presenting in HTS data. The example given was of technical replicates
  which will naturally have lower dispersion rates than true biological
  replicates. I would suggest that the authors include some evidence
  that low dispersion is present across biological replicates if they
  want to make the claim that it is one of the largest sources of HTS
  data pathologies.

  \begin{itemize}
  \tightlist
  \item
    We have now included two biological datasets and modeling data
  \end{itemize}
\item
  In the metatranscriptomic's example it is not entirely clear why
  housekeeping genes would have a non-zero log fold change between the
  groups in the first place. Well the authors suggest that
  misidentification of these genes is due to inappropriate scale
  assumptions, it would help to identify what assumptions made by the
  normalization result in this potential pathology to begin with.

  \begin{itemize}
  \tightlist
  \item
    The reviewer is correct in that HK genes may not have non-0 LFC, but
    that the assumption that is almost always made is that these are
    appropriate references. Note that the analysis in the
    metatranscriptome is not at the gene level, but at the aggregated
    functional level. We have tried to clarify that the difference in
    occurrence is a major driver of this asymmetry. (lines 358-365)
  \end{itemize}
\item
  Could the authors comment on whether they believe their scale model
  implementation will work well under other normalization conditions
  other than CLR.

  \begin{itemize}
  \tightlist
  \item
    Nixon et al.~show that in theory, any interpretable log-ratio will
    work with this schema and other papers are in preparation showing
    this. For the purposes of this manuscript the CLR is the
    normalization used by ALDEx2
  \end{itemize}
\end{enumerate}

Associate Editor: Erb, Ionas Comments to the Author: Dear authors,

Please focus your revision on the following aspects: State what the true
scale of the data should describe, i.e., number of transcripts relative
to species abundance or similar. Does this change between data sets?
Include a comparison with other approaches, e.g., comparing with an
approach that uses DNA abundance as described in the paper in Annual
Reviews (see reviewer 2).

\begin{itemize}
\tightlist
\item
  we have added a paragraph in the discussion about this. Fundamentally,
  while the scale approach is generalizable and could be applied to the
  types of data given in the review paper (the actual method is in (6)),
  but that approach is designed to ask a different type of question than
  is addressed by scale. (Lines 374-380 and above)
\end{itemize}

A comparison with the use of a moderated statistic (see, e.g., ~the
classical Smyth 2004 paper) which also addresses the low dispersion
problem in data with low sample size would add considerable value, even
if only as a discussion.

\begin{itemize}
\tightlist
\item
  we have added some discussion of the moderated approach in the
  discussion. However, fundamentally this approach is trying to solve a
  problem of missing information in the data. The use of scale
  uncertainty directly addresses this issue because it adds in
  uncertainty around the missing information rather than try to model it
  from the data. This is the fundamental problem of partially specified
  models that we bring into the introduction, when the information is
  not available, one is better to acknowledge this rather than to try
  and pull it out of data that is missing it. (lines 51-53)
\end{itemize}

Additionally, the inclusion of a simulated or otherwise known ground
truth (as pointed out by reviewer 2) would validate the claims in Figure
1.

\begin{itemize}
\tightlist
\item
  This has been done with two different datasets and the result has
  significantly strengthend the paper and our interpretations
\end{itemize}

Discuss effect of biological replicates vs.~technical replicates (see
both reviewers).

\begin{itemize}
\tightlist
\item
  This has been added in multiple places and again the introduction of
  the simulated dataset has helped with this point.
\end{itemize}

Please indicate if the preprints referenced in (3) and (20) are
submitted / currently under review.

\begin{itemize}
\tightlist
\item
  The companion articles by by McGovern et al. (1) discussing the use in
  gene set enrichment and by Nixon et al. (2) targeting the microbiome
  community are published in PLoS Comp. Bio. and under second revision
  at Genome Biol. respectively. The companion article introducing the
  theory and the initial modeling by Nixon et al. (4) is still under
  review at Annals of applied Statistics which is a frustration for the
  authors as much as the reviewers
\end{itemize}

\phantomsection\label{refs}
\begin{CSLReferences}{1}{1}
\bibitem[\citeproctext]{ref-McGovern:2023}
1. McGovern,K.C., Nixon,M.P. and Silverman,J.D. (2023)
\href{https://doi.org/10.1371/journal.pcbi.1011659}{Addressing erroneous
scale assumptions in microbe and gene set enrichment analysis}.
\emph{PLoS Comput Biol}, \textbf{19}, e1011659.

\bibitem[\citeproctext]{ref-Nixon2024B}
2. Nixon,M.P., Gloor,G.B. and Silverman,J.D. (2024) Beyond
normalization: Incorporating scale uncertainty in microbiome and gene
expression analysis. \emph{bioRxiv},
\href{https://doi.org/10.1101/2024.04.01.587602}{10.1101/2024.04.01.587602}.

\bibitem[\citeproctext]{ref-Dos-Santos:2024aa}
3. Dos Santos,S.J., Copeland,C., Macklaim,J.M., Reid,G. and Gloor,G.B.
(2024) \href{https://doi.org/10.1186/s40168-024-01992-w}{Vaginal
metatranscriptome meta-analysis reveals functional {BV} subgroups and
novel colonisation strategies}. \emph{Microbiome}, \textbf{12}, 271.

\bibitem[\citeproctext]{ref-nixon2024scale}
4. Nixon,M.P., McGovern,K.C., Letourneau,J., David,L.A., Lazar,N.A.,
Mukherjee,S. and Silverman,J.D. (2024)
\href{https://arxiv.org/abs/2201.03616}{Scale reliant inference}.

\bibitem[\citeproctext]{ref-Wu2021}
5. Wu,J.R., Macklaim,J.M., Genge,B.L. and Gloor,G.B. (2021)
\href{https://doi.org/10.1007/978-3-030-71175-7_17}{Finding the centre:
Compositional asymmetry in high-throughput sequencing datasets}. In
Filzmoser,P., Hron,K., Martìn-Fernàndez,J.A., Palarea-Albaladejo,J.
(eds), \emph{Advances in compositional data analysis: Festschrift in
honour of vera pawlowsky-glahn}. Springer International Publishing,
Cham, pp. 329--346.

\bibitem[\citeproctext]{ref-Zhang:2021ab}
6. Zhang,Y., Thompson,K.N., Huttenhower,C. and Franzosa,E.A. (2021)
\href{https://doi.org/10.1093/bioinformatics/btab327}{Statistical
approaches for differential expression analysis in metatranscriptomics}.
\emph{Bioinformatics}, \textbf{37}, i34--i41.

\end{CSLReferences}

\end{document}
